\chapter{INTRODUCTION}
\pagenumbering{arabic}
\setcounter{page}{1}

This project focuses on Human Computer Interaction (HCI) research specifically involving people with dementia.  We intend to explore the opportunities to design engaging interaction to enable meaning connections between people with dementia and others, addressing the social difficulty dementia creates.
We intend to explore possibilities of using Tangible User Interfaces (TUIs) in a number of prototype entertainment systems to evaluate their effectiveness.

People with dementia suffer from common symptoms such as, memory, language, orientation difficulties and problems with their concentration, planning or organisation abilities. The World Health Organisation defines dementia by the following:

\begin{quote}
\emph{``Dementia is a syndrome due to disease of the brain, usually of a chronic or progressive nature, in which there is disturbance of multiple higher cortical functions, including memory, thinking, orientation, comprehension, calculation, learning capacity, language, and judgment. Consciousness is not clouded. Impairments of cognitive function are commonly accompanied, and occasionally preceded, by deterioration in emotional control, social behaviour, or motivation''~\cite{organizationicd10}.}
\end{quote}

These symptoms lead to people facing feelings of isolation and loneliness, as memory and language difficulties directly impair their social skills and conventional interactions.
This can lead to the feeling of being a burden to people around them such as family members and care workers.
The carers themselves also experience difficulties, as with the increased responsibility comes distress, leading to severe behavioural, cognitive, and functional impairments.
Care workers and others also find it hard to engage with people with dementia due to the afore mentioned social difficulties dementia causes. 
There can be a disconnection between them and the carer, further increasing the feeling of isolation and loneliness for the person and less opportunity for carers to monitor their wellbeing. 

In the UK alone there are over 835,000 people living with dementia as of 2014~\cite{dem-stats}. With these increasing numbers we believe its timely to explore the possibilities for computer assisted entertainment for people living with dementia.
This can serve to calm and even empower people suffering from dementia though what capabilities they still maintain. 
While also intended to give opportunities to care workers to better engage with people with dementia, meaning potential for increased health monitoring and to aid the stress of the job of caring for someone suffering from dementia.

With this we intended to build multiple prototype systems dubbed, \emph{Shared-Symphonies}, \emph{Mingle-Mugs} and \emph{EnviroRoom} each offering different means of interactions and shared experience but all with the intent of evaluating how we can increase social connection between people with dementia, care workers and others.
We believe this will create more opportunities for carers to monitor their condition though increased dialogue and observations of capabilities, aiding carers in monitoring memory and language skills, while building physical interaction will also aid in tracking visuospatial or orientation ability.

\section{Project Aims}
With this motivation outlined our aims will be centred around improving the understanding of the field of Human Computer Interaction (HCI) in relation to people with dementia.
In this project we aim to design and build a system that take into consideration the following aim:
\begin{quote}
\bf{Research and develop TUIs as part of an entertainment system that encourages meaningful group social engagement between people with and without dementia.}
\end{quote}
In order to achieve this aim we have identified the following sub-objectives:
\begin{enumerate}
\item Use of TUIs to better facilitate group interactions and accommodate the design technique cultural probing (explained in more detail in section \ref{probing}).
\item Accommodate for a range of cognitive and physical capabilities, to aid social interactions as everyone will be able to engage in the system, empowering people who face increasing loss of capabilities.
\item Understanding how to design enjoyable experiences that serve to calm and offer therapeutic activity for people living with dementia.
\end{enumerate}

\section{Outline}