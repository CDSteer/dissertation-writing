\chapter{RELATED WORK}\label{lit}
In this section we examine related areas to our project in order to gain clear understanding of what has been done before in order to gain grounding to further develop the field.

\section{Healthcare Benefits}
Before researching the intervention of computers to provide entertainment for people living with dementia. We worked to understand what healthcare benefits this can provide.

Social Interaction provides a clear health benefit for people with dementia it provides meaning and elevates self-esteem~\cite{vernooij2007}.
Systems such as ASSIST(Assistive Story Intervention Technology) and TR-Bios are examples were digital intervention has aided in providing a positive effect in assisting conversation between people with dementia and people without~\cite{Dijkstra2004263, cohen2000two}.

We also see how designers have made systems that make use of peoples remaining capabilities to build self esteem and promote independence\cite{VanRijn2010, Sterns2005}. These design considerations relate to the topic of \emph{Personhood} defined by Tom Kitwood~\cite{personhood} this outlines how we should be looking to person-centred care, a philosophy we can apply though out our project by putting ourselves in the ``shoes'' of the user.
% We also see how we can apply Montessori programming to dementia activities..... ~\cite{jarrott2008montessori}

\section{Existing Entertainment}
There has already been extensive studies that highlighting the benefits of elderly people playing video games as a form of entertainment~\cite{gamberini2008}. Furthermore we see there are many great examples that directly relate to the field of games for people living with dementia, with many already being deployed in user studies with reported results. We present the following information on these systems.

\subsection{Group Games}
Many existing games both digital and non-digital are intended to be played in a group situation~\cite{Sterns2005, VanRijn2010, sterns2008human}.
This correlates with the perviously outlined health benefits of developing social interaction aids. 
One such system is named \emph{`Chitchatters'}, this system gives great insight into the design and deployment meant of entertainment systems for people living with dementia.
In the design phase the team identify three main meaningful activities for people living with dementia, and build the system addressing these needs. These activities include, reminiscing, music and family and social activities~\cite{VanRijn2010}.

\emph{`Chitchatters'} is designed around sparking conversation within groups of people using the system. 
The developers used objects to trigger reminiscent activities and to provide tactile stimulation with feedback of music, video, spoken poetry and items to hold and touch from the past. 
The system consists of four main components, telephone to read the poetry, radio for the music, a treasure box containing the items and a television for the videos~\cite{VanRijn2010}.
Each item is placed around the room, then in a group setting each one lights up as a signifier its time to interact.
Each of these interaction is intended to spark conversation and discussion within the group.
The team draws the following conclusions, using old fashioned looking objects meant the users could engage without needing to learn new gestures, it aids reminiscing therapy. They also make reference to Montessori philosophy and how \emph{`Chitchatters'} serves to elevate ones self-esteem, an important factor for people with limited and increasing loss of capabilities.

Another example of group games is \emph{`Memory Magic'}, this example is non digital~\cite{Sterns2005}, but gives clear advise on the topic of designing games for people with dementia. 
Their motivation is to address the common problem of the boredom people face in care homes. 
They first state how you should avoid childish activities like giving them toys or just busy work. 
They also outline good support for making activities inclusive and to avoid competition, as this could lead to exclusion as people have varying levels of dementia. 
Therefore in group games people with more serious dementia are less capably\emph{`Memory Magic'} aimed to avoid this~\cite{Sterns2005}.
Their game minimises this by shifting the focus to the participation, socialising and encouraging design to take advantage of the capabilities left and prompt independence and overall promote cognitive stimulation and social interaction. 
Results of the user study showed positive engagement and a reduction of negative behaviours, highlighting benefits of group inclusion and democratising ability to interact.

\subsection{Health Monitoring}
With digital intervention we see the door open to possibilities for increased ability and accuracy in heath monitoring. The first example relates closely to our project, it's a game specifically designed for people with dementia, named \emph{`ZPLAY'}. With two versions the \emph{@lab}, designed for diagnostics, and \emph{@home} to challenge people with dementia's engagement~\cite{makedon2010interactive}. 
As people suffering with dementia have different stages of mental functionality, the game could easily be reconfigured to suit different users. 
The \emph{`ZPLAY'} logs the data of the games challenge the users cognitive motor control response. 
The paper concludes \emph{`ZPLAY'} could collect useful information to aid people with dementia gain a level of independency that may improve their quality of life~\cite{makedon2010interactive}.

Another example of a health monitoring system is \emph{`Symbiosis'}. Described as an innovative human computer interaction environment for Alzheimer's support~\cite{symbiosis}. As Alzheimer's decease is a common symptom of dementia the work directly relates to our project. 
This part of the system is called \emph{`SymbioGammers'}, this part of the systems is a suite of games that exercise memory, attention and orientation skill, along with visual and space perception. 
Their games used Microsoft's Kinect Camera\footnote{Kinect: \url{http://www.xbox.com/en-gb/kinect}} as the controller stating how it offered both physical and mental exercise~\cite{symbiosis}.
Of the people with Alzheimer's who took part in their user study, they saw an increase in mental efficiency from interacting with the games as well as their interaction engagement.

\subsection{Music Therapy}
Other research has incorporated music therapy into games specifically developed for people living with dementia~\cite{Boulay2011, symbiosis, Riley2009}. The research has the common goal of providing a therapeutic experience for people with dementia. The \emph{`MINWii'} project utilises Nintendo Wii and aims by design to deliver low cognitive and motor requirements, failure-free game play with simple design, this system also supports our claim of putting human contact first explaining how their music production system facilitated corporative play. The paper begins by outlining the philosophy of Renarcissization is a term used for the process of restoring self-esteem and how music therapy though the medium of video game could facilitate this motive~\cite{Boulay2011}. The paper concludes by stating how healthcare professionals were able to effectively learn the system with no prior training and how the people with dementia demonstrated strong interest in the system, from which we can interpret as enjoying music interaction.

The previously descried system \emph{`Symbiosis'} also has a music therapy component named \emph{`SymbioMusic'}~\cite{symbiosis}. The paper outlines how music can shift mood, manage stress-induced agitation, stimulate positive interactions, facilitate cognitive function, and coordinate motor movements.\emph{`SymbioMusic'} offers a variety of music therapy scenarios, the system uses the Kinect sensor and aims to invoke musical expression combining music revival and memory refreshment. The study reports no direct link to the \emph{`SymbioMusic'} to the increased mental efficiency or engagement instead  \emph{`SymbioGames'} raising questions to the effectiveness of music therapy for enhancing the lives of people with dementia.

Other studies show no direct statistical coloration for quantifiable positive effects from the use of music therapy on people living with dementia~\cite{vink2003}. Through the study \emph{`MINWii'} has shown it can facilitate social group experiences and reminiscing, both factors in creating meaningful experiences for people with dementia. With this we feel its worth while to further explore the use of music in our in entertainment for people with dementia to further evaluate if it has a beneficial impact as a social engagement mechanism.

\section{Design Challenges}
We can see there is great support the intervention of games into healthcare for people living with dementia.
Due to the deterioration of cognitive and motor capabilities in people with dementia, this present certain design challenges.

\subsection{Interface Design}
There has been previous research outlining certain guidelines in game design for people living with dementia~\cite{Bouchard2012} based on interactive prototyping. 
With respect to building graphical output we can take on board a number of guidelines.
These include, keeping scenes simple, use warm and bright colours with simple textures, create good luminosity but avoid dazzling the player, clearly defined contrasts to improve depth perception~\cite{Bouchard2012}.
It also outlines how we keep trace of patient's cognitive abilities and keep the player in their ``flow zone'' with respect to the gameplay.
Final as we are working with TUIs guides uses of light weight interfaces is useful point to take on board~\cite{Bouchard2012}.
Overall we should intend to follow, evaluate and extend these guidelines during the development of our systems.

\subsection{Cultural Probes}\label{probing}
Cultural Probes are used to gather information about participants culture, thought and values, often though the use of an array of objects, that record events, feeling and interactions~\cite{brown2014, gaver1999}. 
As previously discussed \emph{`Chitchatters'} made good use of cultural probing being able to offer reminiscing therapy. 
The better we understand our users culture, thought and values the more meaningful experiences we can offer.

\section{Tangible User Interfaces}
After evaluation of related work discussed in pervious sections we see an overwelming support for building collaborative group interactions, to simulate the health benefit of social engagement and elevating self-esteem. We believe exploring and evaluating the use of TUIs in building entertainment systems could further advance group user engagement and enhance user experience.

Defined in the HCI community in 1997 a tangible user interfaces (TUI) is an interfaces where the user interacts with digital information through the physical environment~\cite{Ishii1997}. We also see there have been studies that support the use of tangible interfaces for group interaction~\cite{Guia2013, Hornecker2006, Lissermann2014, Taylor2008, Taylor2011}. 
As previously discussed, group social interactions have a positive effect on people with dementia. 
So with the given support for tangible interfaces affording for group interactions we think it will be worthwhile exploring and evaluating the use of TUIs for people living with dementia. 

As perviously outlined interaction with music has potential for offering therapeutic experiences for people with dementia~\cite{Boulay2011, symbiosis, Riley2009}. Though the pervious work in this has not explicitly introduced a group interface for group interaction with music. There have been studies researching tangible interaction with music within a group not directly intended for people with dementia~\cite{Taylor2008, Taylor2011}. We intend to extend and explore how this can be applied to the design of group music interfaces for people with dementia.    

% Humanaquarium\cite{Taylor2011}
% \\ Dream.Medusa\cite{Taylor2008}

Another motive for the use of TUI is due to the complexity the standard combination of mouse and keyboard with a screen presents to users with cognitive deficits~\cite{boussemart2007}.
Other advantages included the ability for them to be build on top of everyday items aiding cultural probing, reminiscing therapy and less stress on the need to learn new gestures, or add more complex technology to their lives~\cite{boussemart2007,Spreicer2012}.

A study researching cognitive rehabilitation used a tangible games interface.
A prototype was developed using NFC tags on a mat of images, each image on the mat has its own NFC tag inside, that the user moves their mobile device close to in order to activate, the interaction was described as easy and intuitive for elderly people in contrast to mouse and keyboard interactions~\cite{Guia2013}. 
This highlights to us how the development of new TUIs offer the ability to go beyond traditional control methods.
Instead giving opportunity to create unique interactions and interfaces specifically targeted at people with dementia

\subsection{TUI Materials}
There is limited studies into the types of materials we should use in the TUIs for people with dementia. Of most relevance is the remark from the \emph{`Chitchatters'} paper which informs us that people with dementia prefer cuddly furry animals rather than non-furry animal robots~\cite{VanRijn2010}. The question of what materials effect the user experience of our systems should be thoroughly evaluated as the project develops. 

The furry robot referenced in the \emph{`Chitchatters'} paper is named \emph{`PARO'}. In a study, the use of \emph{`PARO'} in sensory groups of people with dementia, interviews with therapists highlighted how it was not possible for all the participants to interact with \emph{`PARO'} simultaneously but still went on to state it's a good social mediator. 
After observing both individual and group interaction it was concluded that \emph{`PARO'} was more appropriate for one by one interaction rather than large groups.

\subsection{Conclusion}
In this section we thoroughly examined existing research and systems that relate to the aims the project. 
Information in these papers has given an insight into what health benefits we should be mindful of providing and we can design and facilitate this. 
But no research has directly evaluated the effectiveness of new TUIs to facilitate group interaction between people with dementia.